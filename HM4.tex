\documentclass{article} 
\usepackage{amsmath}
\usepackage{amssymb}
\usepackage{fancyhdr}
\pagestyle{fancy}
\lhead{} 
\chead{} 
\lhead{\bfseries Group: SHPZKN}
\rhead{\bfseries ML Homework 4} 

\newcommand{\Lagr}{\mathcal{L}}

\begin{document}
\section{Problem 1}
%%%a--------------------------------------------------------------------------------------------------------------------------------------------------------
a)\\
we formulate the Lagrangian function:
\begin{flalign*}
\begin{split}
\Lagr(\theta, \lambda) &= \sum\limits_{k= 1}^{n} ||\theta-x_k||^2 + \lambda(\theta^\intercal b-0)= \sum\limits_{k= 1}^{n} (\theta - x_k)^\intercal(\theta - x_k) + \lambda \theta^\intercal b\\
&=\sum\limits_{k= 1}^{n} (\theta^\intercal \theta + x_k^\intercal x_k - 2\theta^\intercal x_k)+ \lambda\theta^\intercal b\\
&= n \theta^\intercal\theta + \sum\limits_{k= 1}^{n} x_k^\intercal x_k- 2 \sum\limits_{k= 1}^{n} \theta^\intercal x_k + \lambda\theta^\intercal b\\
\frac{\partial \Lagr(\theta, \lambda)}{\partial \theta} &= 2n\theta-2\sum\limits_{k=1}^{n}x_k+\lambda b \stackrel{!}{=}0\\
\theta^* &=\frac{2\sum\limits_{k=1}^{n}x_k+  \lambda b}{2n} = \bar x + \frac{\lambda b}{2n}\\ \\
\frac{\partial \Lagr(\theta, \lambda)}{\partial \lambda} &=  \theta^\intercal b \stackrel{!}{=}0\\
&\left(\bar x + \frac{\lambda b}{2n} \right)^\intercal b = 0\\
\lambda &= -2n\bar x ^\intercal b (b^\intercal b )^{-1}\\
\text{plug $\lambda$ into the formular of $\theta^*$:}\\
\theta^* & = \bar x - \bar x ^\intercal b(b^\intercal b)^{-1}b\\
%\frac{\partial ^2}{\partial \lambda^2} \Lagr(\theta, \lambda)&= 0  \qquad  \frac{\partial ^2}{\partial \theta^2} \Lagr(\theta, \lambda)= 2n \qquad 
%\frac{\partial ^2}{\partial \theta \partial \lambda} \Lagr(\theta, \lambda)= 0
\end{split}&
\end{flalign*}
\textbf{Geometrical interpretation:} The new minimizing parameter $\theta^*$ is the linear combination of the unconstrained parameter $\theta$, which means the value of $\theta^*$ is influenced by $b$ as well, the special case is that when b is a scaler, $\theta^*$ will equal to zero.
\clearpage

%%%b--------------------------------------------------------------------------------------------------------------------------------------------------------
b)\\\
we formulate the Lagrangian function:
\begin{flalign*}
\begin{split}
\Lagr(\theta, \lambda) &= \sum\limits_{k= 1}^{n} ||\theta-x_k||^2 + \lambda(||\theta-c||^2-1) 
= \sum\limits_{k= 1}^{n} (\theta - x_k)^\intercal(\theta - x_k) +\lambda[(\theta-c)^\intercal(\theta-c)-1]\\
&=\sum\limits_{k= 1}^{n} (\theta^\intercal \theta + x_k^\intercal x_k - 2\theta^\intercal x_k)+ \lambda(\theta^\intercal \theta + c ^\intercal c - 2 \theta^\intercal c -1)\\
&=n\theta^\intercal \theta+\sum\limits_{k= 1}^{n} x_k^\intercal x_k- 2 \sum\limits_{k= 1}^{n} \theta^\intercal x_k +  \lambda(\theta^\intercal \theta + c ^\intercal c - 2 \theta^\intercal c -1)\\
\frac{\partial \Lagr(\theta, \lambda)}{\partial \theta} &= 2n\theta-2\sum\limits_{k=1}^{n}x_k+\lambda (2 \theta -2 c) \stackrel{!}{=}0\\
\theta^* &=\frac{\sum\limits_{k=1}^{n}x_k + \lambda c}{n + \lambda} = \frac{n\bar{x}+\lambda c}{n + \lambda}\\
\frac{\partial \Lagr(\theta, \lambda)}{\partial \lambda} %&= \theta^\intercal \theta + c ^\intercal c - 2 \theta^\intercal c -1 
& = ||\theta -c||^2\stackrel{!}{=}0 \qquad \text{ plug $\theta^*$ into the euquation}\\
&=  \left |\left| \frac{n\bar{x}+\lambda c}{n + \lambda}-c\right|\right|^2
=  \left |\left| \frac{n\bar{x}+\lambda c}{n + \lambda}-\frac{c(n+ \lambda)}{n 
+\lambda}\right|\right|^2 = n^2 \left |\left| \frac{\bar{x}-c}{n + \lambda}\right|\right|^2 = 1\\
&(n+ \lambda)^2 =n^2||\bar{x}-c||^2\\
&\lambda^* = \pm n ||\bar{x}-c||-n \qquad \text{ plug $\lambda^*$ into the euquation of $\theta^*$}\\
\theta^*&= \frac{n\bar{x}+(\pm n ||\bar{x}-c||-n )c}{n \pm  n ||\bar{x}-c||-n}\\
&= \frac{\bar{x}-c\pm||\bar{x}-c||}{\pm  ||\bar{x}-c||}\\
&= \frac{\bar{x}-c}{||\bar{x}-c||}\pm c
\end{split}&
\end{flalign*}
\\
\\
\textbf{Geometrical interpretation:} The new $\theta^*$ is shifted by $c$ and normalized, it's restricted to a circle with center $c$ and radius 1.
\\
\\
\section{Problem 2}
%%%a--------------------------------------------------------------------------------------------------------------------------------------------------------
a)\\
We know that if $A$ is an $n\times n$ matrix and let $\lambda_1 \text{,..,}\lambda_n$ be its eigenvalues. Here $det(A)$ is the determinant of the matrix $A$ and $tr(A)$ is the trace of the matrix $A$, and the determinant of $A$ is the product of its eigenvalues, and  the trace of $A$ is the sum of the eigenvalues. Hence, the trace of $S$ equals to the sum of eigenvalues, and $S_ii$ is the diagonal elements of the scatter matrix, therefore:
$$tr(S) = \sum\limits_{i =1}^{d} \lambda_i =  \sum\limits_{i =1}^{d} S_{ii} $$
It holds that $\lambda_i \geq 0, \forall i=1,...,d$, since $S$ is positive semi-definite,
$$\lambda_1 \leq \sum\limits_{i =1}^{d} \lambda_i  =  \sum\limits_{i =1}^{d} S_{ii} $$
%%%a--------------------------------------------------------------------------------------------------------------------------------------------------------
b)\\
It holds that $\lambda_1 =  \sum\limits_{i =1}^{d} S_{ii} $ if and only if $\lambda_i =0, \forall i=2,...,d$.
In this case the matrix $S$ is of rank 1, which means that all features are linearly dependent.
\\
\\
%%%c--------------------------------------------------------------------------------------------------------------------------------------------------------
c)\\\
we could rewrite the matrix $\mathbf{S}$ on below,
$$\mathbf{S} =  \mathbf{T} \mathbf{\Lambda}\mathbf{T}^\intercal$$
the matrix $\mathbf{T}$ contains the eigenvectors $\mathbf{\tau_1}, ..., \mathbf{\tau_d}$ and $ \mathbf{T}^\intercal \mathbf{T} = \mathbf{I}_d$. the matrix $\mathbf{\Lambda} = diag(\lambda_1,..., \lambda_d)$ contains the eigenvalues, and $\lambda_1$ is the biggest eigenvalue.\\
\\
For any vector $\nu$, and $||\nu||=1$ we rewrite:
$$\mathbf{\nu}^\intercal\mathbf{S}\mathbf{\nu} = \mathbf{\nu}^\intercal \mathbf{T} \mathbf{\Lambda}\mathbf{T}^\intercal \mathbf{\nu}$$
we define $\mathbf{\omega} =  \mathbf{\nu}^\intercal \mathbf{T}$,
\begin{flalign*}
\begin{split}
\mathbf{\nu}^\intercal\mathbf{S}\mathbf{\nu} &=  \mathbf{\nu}^\intercal \mathbf{T} \mathbf{\Lambda}\mathbf{T}^\intercal \mathbf{\nu} = \mathbf{\omega}^\intercal \mathbf{\Lambda} \mathbf{\omega}\\
&= \lambda_1 \omega_1^2 +...+ \lambda_d\omega_d^2\\
&\leq \lambda_1(\omega_1^2+...+ \omega_d^2)\\
&= \lambda_1||\omega||^2 = \lambda_1 \mathbf{\omega}^\intercal \mathbf{\omega}\\
&=  \lambda_1 \mathbf{\nu}^\intercal \mathbf{T} \mathbf{T}^\intercal  \mathbf{\nu} 
= \lambda_1 \mathbf{\nu}^\intercal\mathbf{\nu} \\
&= \lambda_1||\nu||^2 = \lambda_1 = \lambda_{max}
\end{split}&
\end{flalign*}
Therefore, if we let $\nu = \tau_i$,
$$\mathbf{S_{ii}} = \tau_i ^\intercal \mathbf{S}\tau_i\leq \lambda_1$$
for any $i=1, ... , d$, which implies $\max_{i=1}^{d}\mathbf{S}\leq\lambda_1$
%%%a--------------------------------------------------------------------------------------------------------------------------------------------------------
\\
\\
d)\\\
The conditon is that $\mathbf{S}_{ii} = 0$, which means the all the upper right and lower left elements of $\mathbf{S}$ are zero.
 


\section{Problem 3}
%%%a--------------------------------------------------------------------------------------------------------------------------------------------------------
a)\\
we formulate the formulas from question:
\begin{flalign*}
\begin{split}
J(\mathbf{w}) &= ||\mathbf{S} \mathbf{w}||-\frac{1}{2} \mathbf{w}^\intercal \mathbf{S} \mathbf{w}\\
\mathbf{v} &= \mathbf{S}^{\frac{1}{2}} \mathbf{w} \qquad \mathbf{w} = \mathbf{S}^{-\frac{1}{2}} \mathbf{v} \\
J(\mathbf{w}) &= ||\mathbf{S} \mathbf{S}^{-\frac{1}{2}} \mathbf{v}||
-\frac{1}{2} (\mathbf{S}^{-\frac{1}{2}} \mathbf{v})^\intercal \mathbf{S} \mathbf{S}^{-\frac{1}{2}} \mathbf{v}\\
&= ||\mathbf{S}^{\frac{1}{2}} \mathbf{v}||-\frac{1}{2} \mathbf{v}^\intercal \mathbf{v}\\
&= ||\mathbf{S}^{\frac{1}{2}} \mathbf{v}||-\frac{1}{2} ||\mathbf{v}||^2\\
\frac{\partial J(\mathbf{v})}{\partial \mathbf{v}} &= \frac{\mathbf{S}^{\frac{1}{2}} \mathbf{v}}
{||\mathbf{S}^{\frac{1}{2}} \mathbf{v}||}\mathbf{S}^{\frac{1}{2}} - \mathbf{v}\\
\mathbf{v} &\gets \mathbf{v} + \gamma \frac{\partial J}{\partial \mathbf{v}} 
=  \mathbf{v} + \gamma  \frac{\mathbf{S}^{\frac{1}{2}} \mathbf{v}}
{||\mathbf{S}^{\frac{1}{2}} \mathbf{v}||}\mathbf{S}^{\frac{1}{2}} - \mathbf{v}\\
\mathbf{v} &\gets \gamma  \frac{\mathbf{S} \mathbf{v}}
{||\mathbf{S}^{\frac{1}{2}} \mathbf{v}||}\\
&\text{If $\gamma$ is identity matrix}\\
\mathbf{S}^{-\frac{1}{2}} \mathbf{v}&\gets \mathbf{S}^{-\frac{1}{2}}   \frac{\mathbf{S} \mathbf{v}}
{||\mathbf{S}^{\frac{1}{2}} \mathbf{v}||}  =  \frac{\mathbf{S}^{\frac{1}{2}} \mathbf{v}}
{||\mathbf{S}^{\frac{1}{2}} \mathbf{v}||}  \\
\mathbf{w} &\gets \frac{\mathbf{S}^{\frac{1}{2}} \mathbf{S}^{\frac{1}{2}} \mathbf{w}}{||\mathbf{S}^{\frac{1}{2}}  \mathbf{S}^{\frac{1}{2}} \mathbf{w}||} \\
\mathbf{w} &\gets \frac{\mathbf{S} \mathbf{w}}{||\mathbf{S}\mathbf{w}||} \\
\end{split}&
\end{flalign*}
\\
\\
b)
$$||\mathbf{w}|| = \left |\left|\frac{\mathbf{S} \mathbf{w}}{||\mathbf{S}\mathbf{w}||} \right|\right| =  \frac{||\mathbf{S} \mathbf{w}||}{||\mathbf{S}\mathbf{w}||} = 1$$












\end{document}